\documentclass[12pt]{scrartcl}



\setlength{\parindent}{0pt}
\setlength{\parskip}{.25cm}

\usepackage{graphicx}

\usepackage{xcolor}

\definecolor{darkred}{rgb}{0.5,0,0}
\definecolor{darkgreen}{rgb}{0,0.5,0}
\usepackage{hyperref}
\hypersetup{
  letterpaper,
  colorlinks,
  linkcolor=red,
  citecolor=darkgreen,
  menucolor=darkred,
  urlcolor=blue,
  pdfpagemode=none,
  pdftitle={CS1 - Lab 13.0 - Java},
  pdfauthor={Christopher M. Bourke},
  pdfkeywords={}
}

\definecolor{MyDarkBlue}{rgb}{0,0.08,0.45}
\definecolor{MyDarkRed}{rgb}{0.45,0.08,0}
\definecolor{MyDarkGreen}{rgb}{0.08,0.45,0.08}

\definecolor{mintedBackground}{rgb}{0.95,0.95,0.95}
\definecolor{mintedInlineBackground}{rgb}{.90,.90,1}

%\usepackage{newfloat}
\usepackage[newfloat=true]{minted}
\setminted{mathescape,
               linenos,
               autogobble,
               frame=none,
               framesep=2mm,
               framerule=0.4pt,
               %label=foo,
               xleftmargin=2em,
               xrightmargin=0em,
               startinline=true,  %PHP only, allow it to omit the PHP Tags *** with this option, variables using dollar sign in comments are treated as latex math
               numbersep=10pt, %gap between line numbers and start of line
               style=default, %syntax highlighting style, default is "default"
               			    %gallery: http://help.farbox.com/pygments.html
			    	    %list available: pygmentize -L styles
               bgcolor=mintedBackground} %prevents breaking across pages
               
\setmintedinline{bgcolor={mintedBackground}}
\setminted[text]{bgcolor={mintedBackground},linenos=false,autogobble,xleftmargin=1em}
%\setminted[php]{bgcolor=mintedBackgroundPHP} %startinline=True}
\SetupFloatingEnvironment{listing}{name=Code Sample}
\SetupFloatingEnvironment{listing}{listname=List of Code Samples}


\title{CSCE 155 - Java}
\subtitle{Lab 13.0 - Searching \& Sorting}
\author{~}
\date{~}

\begin{document}

\maketitle

\section*{Prior to Lab}

Before attending this lab:
\begin{enumerate}
  \item Read and familiarize yourself with this handout.
  \item Review notes on Search \& Sorting
\end{enumerate}

Some additional resources that may help with this lab:
\begin{itemize}
  \item Wikipedia Article on Selection Sort: \\
  	\url{http://en.wikipedia.org/wiki/Selection_sort}
  \item A Java implementation of Selection Sort:\\
  	\url{http://en.literateprograms.org/Selection_sort_%28Java%29}
  \item API Specification for \mintinline{java}{Arrays}: 
	\url{http://docs.oracle.com/javase/6/docs/api/java/util/Arrays.html}
\end{itemize}

\section*{Peer Programming Pair-Up}

To encourage collaboration and a team environment, labs will be
structured in a \emph{pair programming} setup.  At the start of
each lab, you will be randomly paired up with another student 
(conflicts such as absences will be dealt with by the lab instructor).
One of you will be designated the \emph{driver} and the other
the \emph{navigator}.  

The navigator will be responsible for reading the instructions and
telling the driver what to do next.  The driver will be in charge of the
keyboard and workstation.  Both driver and navigator are responsible
for suggesting fixes and solutions together.  Neither the navigator
nor the driver is ``in charge.''  Beyond your immediate pairing, you
are encouraged to help and interact and with other pairs in the lab.

Each week you should alternate: if you were a driver last week, 
be a navigator next, etc.  Resolve any issues (you were both drivers
last week) within your pair.  Ask the lab instructor to resolve issues
only when you cannot come to a consensus.  

Because of the peer programming setup of labs, it is absolutely 
essential that you complete any pre-lab activities and familiarize
yourself with the handouts prior to coming to lab.  Failure to do
so will negatively impact your ability to collaborate and work with 
others which may mean that you will not be able to complete the
lab.  

\section{Lab Objectives \& Topics}
At the end of this lab you should be familiar with the following
\begin{itemize}
  \item Understand basic searching and sorting algorithms
  \item Understand how comparators work and their purpose
  \item How to leverage standard search and sort algorithms built into a framework
\end{itemize}

\section{Background}

Recursion has been utilized in several searching and sorting 
algorithms. In particular, quicksort and binary search both can 
be implemented using recursive functions.  The quicksort 
algorithm works by choosing a pivot element and dividing a 
list into two sub-lists consisting of elements smaller than the 
pivot and elements that are larger than the pivot (ties can be 
broken either way).  A recursive quicksort function can then 
be recursively called on each sub-list until a base case is 
reached: when the list to be sorted is of size 1 or 0 which, 
by definition, is already sorted.

Binary search can also be implemented in a recursive manner.  
Given a sorted array and a key to be searched for in the array, 
we can examine the middle element in the array.  If the key is 
less than the middle element, we can recursively call the binary 
search function on the sub-list of all elements to the left of the 
middle element.  If the key is greater than the middle element, 
we recursively call the binary search function on the sub-list of 
all elements to the right of the middle element.

Searching and sorting are two solved problems.  That is, most 
languages and frameworks offer built-in functionality or 
standard implementations in their standard libraries to facilitate 
searching and sorting through collections.  These implementations 
have been optimized, debugged and tested beyond anything that 
a single person could ever hope to accomplish.  Thus, there is 
rarely ever a good justification for implementing your own 
searching and sorting methods.  Instead, it is far more important 
to understand how to leverage the framework and utilize the 
functionality that it already provides.  

\section{Activities}

The activities in this lab will involve the same baseball data as 
used in a prior lab.  A complete framework has been provided 
to you to load data on the 2011 National League teams.  There 
is more data this time around and we have defined a structure 
to encapsulate team data as well as several functions to process 
the input file and construct instances of the \mintinline{c}{Team} 
structures for you.

Clone the project code for this lab from GitHub using the following 
URL: \url{https://github.com/cbourke/CSCE155-Java-Lab13}.

\subsection{Sorting the Wrong Way}

In this activity, you are to implement a selection sort algorithm to 
sort an array of \mintinline{java}{Team} objects.  Refer to lecture notes, 
your text, or any online source on how to implement the selection sort 
algorithm.  The order by which you will sort the array will be according 
to the total team payroll (in dollars) in increasing order.

\subsubsection*{Instructions}

\begin{enumerate}
  \item Familiarize yourself with the \mintinline{java}{Team} class and 
	the methods provided to you (the \mintinline{java}{main} method 
	automatically loads the data from the data file and provides an 
	array of teams).  
  \item Implement the \mintinline{java}{selectionSortTeamsByPayroll} 
	function in the \mintinline{text}{MLBTeamUtils.java} file as specified
  \item Run your program
\end{enumerate}

\subsection{Slightly Better Sorting}

The \mintinline{java}{Team} class has many different data fields; 
wins, losses (and win percentage), team name, city, state, payroll, 
average salary.  Say that we wanted to re-sort teams according to 
one of these fields in either ascending or descending order (or even 
some combination of fields--state/city for example).  Doing it the 
wrong way as above we would need to implement no less than 
16 different sort functions!

Most of the code would be the same though; only the criteria on 
which we update the index of the minimum element would differ.  
Instead, it would be better to implement one sorting function and 
make it configurable: provide a means by which the function can 
determine the relative ordering of any two elements.  The way of 
doing this in Java is to define a \emph{comparator} class.

\mintinline{java}{Comparator} classes are simple.  They are required 
to implement the interface \mintinline{java}{Comparator<T>}, which 
means specifying (coding) one method: \mintinline{java}{compare(T a, T b)}.  
The \mintinline{java}{T} in this declaration is a generic that we specify 
when we create a comparator.  The method itself takes two arguments: 
$a$ and $b$ (of type \mintinline{java}{T}) and returns:
\begin{itemize}
  \item A negative value if $a < b$ 
  \item Zero if $a$ is equal to $b$
  \item A positive value if $a > b$
\end{itemize}

For example, in order to sort an array containing references to 
instances of \mintinline{java}{Team} by name, we would create 
a class that implements \mintinline{java}{Comparator<Team>} and 
implement the \mintinline{java}{compare()} method to compare two 
references to \mintinline{java}{Team}s passed in, and return an 
integer value based on the above rules.

Several completed comparator classes have been provided as 
examples.  All the comparator classes take two \mintinline{java}{Team} 
arguments, and return an integer based on the specified comparison.  
Refer to the provided comparator classes for concrete examples.  

\subsubsection*{Instructions}

\begin{enumerate}
  \item Implement the \mintinline{java}{PayrollDescendingComparator}
  	class.  Only one method, \mintinline{java}{compare()} needs
	to be coded.
  \item Use the completed comparator classes provided with this lab 
  	an example on how to implement the \mintinline{java}{compare()} 
	method 
  \item Implement the \mintinline{java}{selectionSortTeams()} method in 
	\mintinline{text}{MLBTeamsUtils.java} source file
  \item Use the comparator class you just created to call 
	\mintinline{java}{selectionSortTeams()} to sort an array of \mintinline{java}{Team}s 
	by payroll descending
\end{enumerate}
	
\subsection{Sorting the Right Way}

The best way to sort is to leverage a sorting algorithm provided by 
a standard library.  This eliminates the need to ``reinvent'' the wheel 
and avoids debugging, testing, etc. of our own code.  Java provides 
several sorting methods in its standard developer kit (SDK).  The 
one we will make use of is in the Arrays class.  The method is able 
to sort arrays containing any type of object by making use of a 
provided comparator:

\mintinline{java}{public static <T> void sort(T[] a, Comparator<? super T> c)}

where
\begin{itemize}
  \item \mintinline{java}{T[]} a is the array of objects (of type 
  	\mintinline{java}{T}) to be sorted
  \item \mintinline{java}{Comparator<? super T> c} is the comparator 
	class that is used to determine sort order of the object elements 
	in the array.  Note: the \mintinline{java}{? super T} in the comparator 
	angle brackets indicate that the type of class being compared must 
	be \mintinline{java}{T}, or any superclass (ancestor) of type \mintinline{java}{T}.
\end{itemize}
	
\subsubsection*{Instructions}

\begin{enumerate}
  \item Examine the source files and observe how \mintinline{java}{Comparator} 
  	classes are defined and how the \mintinline{java}{Arrays.sort} method is called
  \item Implement two more comparators, \mintinline{java}{StateComparator} 
	which sorts based on team home state, and \mintinline{java}{StateCityComparator} 
	which sorts based first on state then by city
  \item Use your method in the \mintinline{java}{main} method to re-sort the array 
	and print out the results.
\end{enumerate}
	
\subsection{Searching}

The \mintinline{java}{Arrays} class also offers a method to search an array.  
In this exercise, we will focus on two strategies for searching--linear search 
and binary search.
\begin{itemize}
  \item \mintinline{java}{linearSearchMLB} - a linear search implementation 
  	(in the \mintinline{java}{MLBTeamUtils} class) that does not require the 
	array to be sorted.  The method works by iterating through the array 
	and calling your comparator class on the provided key (an object instance 
	whose state matches the criteria that you are searching for).  It returns
	the index of the first instance such that the comparator returns zero for 
	(indicating equality between the objects).
  \item \mintinline{java}{binarySearch} - a binary search implementation (in the 
	\mintinline{java}{Arrays} class) that requires the array to be sorted according 
	to the same comparator used to search.  
\end{itemize}

Both methods require a comparator (as used in sorting) and a key upon 
which to search.  A key is a ``dummy'' instance of the same type as the 
array that contains values of fields that you?re searching for.

\subsubsection*{Instructions}

\begin{enumerate}
  \item Examine the \mintinline{java}{linearSearchMLB} method in the 
  	\mintinline{java}{MLBTeamUtils} class and understand how it works
  \item Answer the questions in your worksheet regarding this code segments
  \item Based on your observations add code to search the array for the team 
	representing the Chicago Cubs:
  \begin{enumerate}
    \item Create a dummy \mintinline{java}{Team} key for the Cubs by instantiating 
  	a dummy team and using empty strings and zero values except for 
	the team name (which should be \mintinline{java}{"Cubs"})
    \item Sort the array by team name using the appropriate comparator
    \item Call the \mintinline{java}{binarySearch} method using your key and 
	the appropriate comparator to find the \mintinline{java}{Team} representing 
	the Chicago Cubs
    \item Print the team to the standard output
    \item Demonstrate your working program to a lab instructor
  \end{enumerate}
\end{enumerate}

\section{Handin/Grader Instructions}

\begin{enumerate}
  \item Hand in \emph{all} of your Java source files and your
  completed \mintinline{text}{worksheet.md}
  through the webhandin (\url{https://cse-apps.unl.edu/handin}) 
  using your cse login and password.  
  \item Even if you worked with a partner, you \emph{both} should
  turn in all files.
  \item Verify your program by grading yourself through the
  webgrader (\url{https://cse.unl.edu/~cse155h/grade/}) using the
  same credentials.
  \item Recall that both expected output and your program's output
  will be displayed.  The formatting may differ slightly which is fine.
  As long as your program successfully compiles, runs and outputs 
  the \emph{same values}, it is considered correct.
\end{enumerate}

\section{Advanced Activities (Optional)}

Selection sort is a quadratic sorting algorithm, thus doubling the input size 
($n$ elements to $2n$ elements) leads to a blowup in its execution time 
by a factor of 4.  Quick sort requires only $n\log(n)$ operations on average, 
so doubling the input size would only lead (roughly) to a blowup in execution 
time by a factor of about 2.  Verify this theoretical analysis by setting up an 
experiment to time each sorting algorithm on various input sizes of randomly 
generated integer arrays.  Use \mintinline{java}{System.nanoTime()} to 
record the elapsed time of your algorithms.
	
\end{document}
